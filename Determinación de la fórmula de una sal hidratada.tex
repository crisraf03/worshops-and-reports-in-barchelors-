 \documentclass{article}
\usepackage[spanish]{babel}
\usepackage[utf8]{inputenc}
\usepackage[T1]{fontenc}
\usepackage{amsmath,amsfonts,amssymb}
\usepackage[dvips]{graphicx}
\usepackage[left=4cm, right=4cm, top=2cm, bottom=2cm]{geometry}
\graphicspath {{imagenes/}}
\usepackage{mathpazo}
\usepackage{float}
 
\title{Práctica Nº9 Determinación de la fórmula de una sal hidratada}
\author{Mendoza C., Martinez V., Pizarro H. \\Laboratorio de Fundamentos de Química \\Universidad del Atlántico}
\date{Martes, 16 de mayo del 2017}

\begin{document}

\maketitle 


\section{Introducción}

Tanto el cálculo de las fórmulas empíricas de un  compuesto, como la deshidratación de sales es un requisito fundamental en la práctica de un estudiante de química. Con el fin de dar a conocer dichos conceptos por medio de la práctica se reprodujo la siguiente experiencia en la cual se espera hacer observación del fenómeno de la deshidratación del $Cu(SO_{4})\cdot 5H_{2}O$  a una escala macroscópica y en base a ella hacer inferencias en el proceso y tratar de generaliza para cualquier tipo de sal con carácteristicas similares.

\section{Objetivos}

\begin{itemize}
\item {Determinar el porcentaje de agua en un hidrato conocido de una sal}
\item{Expulsar el agua de una sal hidratada por descomposición}
\end{itemize}

\section{Procedimiento}

Una vez lavados y secados los materiales de trabajo. se procedió a: 

\begin{itemize}
\item pesado de tubo de ensayo y el montaje del experimento.
\item se agrego la sal sobre en tubo de ensayo y se peso nuevamente. 
\item se inicio a calentar la muestra, la cual desprendió agua en forma de vapor.
\item Una vez la sal torno a color blanco se dejo de calentar y se peso nuevamente.
\end{itemize}


\section{Observaciones}
 Durante el proceso de calentado se observo un progreso paulatino, los cambios observados se enuncian a continuación:
\begin{itemize}
 \item la muestra desprendió un líquido amarillento
 \item la muestra se torno de un color opaco
 \item la muestra se torno anaranjada por la parte de abajo
 \item prosigo a tornarse de un color verde claro
 \item la muestra se torno blanca y se precipito agua por el tubo lateral. (no más de un gramo)
\end{itemize}
 posterior a ello, una vez enfriada y pesada la muestra se agregó el agua destilada al residuo de la muestra calentada, la muestra se torno azul nuevamente y se calentó. Hubo partes que se mantuvieron de color blanco, junto el fondo que se persistió de color amarillo.

\section{Datos}


Los datos recogidos y observados en el laboratorio fueron lo siguientes: 

\begin{table} [H]
\centering
\begin{tabular}{|c|c|}
\hline 
\rule[-1ex]{0pt}{2.5ex} peso del tubo de ensayo vacio & 22,06 g \\ 
\hline 
\rule[-1ex]{0pt}{2.5ex} peso del tubo de ensayo + sal & 23,97 g \\ 
\hline 
\rule[-1ex]{0pt}{2.5ex} peso del tubo de ensayo después del calentado & 23,32 g  \\ 
\hline 
\rule[-1ex]{0pt}{2.5ex} cambio de color al calentar la sal hidratada & blanco  \\ 
\hline 
\rule[-1ex]{0pt}{2.5ex} cambio de color al agregar la sal de hidratación & azul \\ 
\hline 
\rule[-1ex]{0pt}{2.5ex} cambio de temperatura  & aumento \\ 
\hline 
\end{tabular} 
\caption{masados y observaciones en el proceso de deshidratación de la sal}
\end{table}

\section{Preguntas}

\begin{enumerate}

\item 
Para calcular el peso de la sal anhidrida se procedió de la siguiente manera:
$$Peso de sal anhidrida = peso tubo de ensayo luego de calentar - peso tubo de ensayo vacio $$
$$peso de sal anhidrida= 23,32g -22,06 g = 1,26 g$$

El nombre de la sal anhidrida es, %pon la formula aca% 

para calcular número de moles en cada producto se procedió de la forma:
$$n_{sal}=\frac{masa sal}{peso molecular} =  \frac{1,26 g}{159.611 g/mol}= 0,008 mol de Cu(SO_{4}) $$

queriendo hallar la masa de agua del compuesto,
$$ masa de agua= masa sal hidratada- masa sal anhidrida = ((23,97-22,06)- 1,26) = 0,65 g de H_{2}O$$
$$n_{agua}=\frac{masa agua}{peso molecular} = \frac{0,65 g}{18 g/mol} = 0,04 moles de H_{2}O$$

el porcentaje de agua en el compuesto esta dado por: 

$$ \% H_{2}O = \frac{n_{agua}}{n_{agua}+n_{sal}}$$
$$ \% H_{2}O = \frac{0,04}{0,04+ 0,08} \times 100 = 33.33\% $$


\item el color de la sal hidratada observada en azul y el de la sal anhídrida blanco.

\item si se calienta fuertemente la sal se observa que torna a amarilla y se adhiere al tubo de ensayo.

\item Si, en la mayoría de los casos se procede sobre el supuesto de que la reacción se presenta en medio de una solución acuosa. Dando por entender que hay agua en el medio, la cantidad de agua en los reactivos puede ser incluida dentro de esta generalidad.



\end{enumerate}

\section{Conclusiones}

\begin{itemize}
 \item Tanto la deshidratación como la hidratación de una sal por medio de calentamiento a simple vista son procesos reversibles
 \item si bien el agua esta presente en una muestra de $Cu(SO_{4})\cdot 5H_{2}O$ o cualquier otra sal, no hace parte de la confuguración del compuesto, siendo esta más un a mezcla que una sustancia pura.
 \item Se atribuye la diferencia del número de moles del agua téorico y experimental al sobercalentado y errores sistematicos durante el procedimiento.
\end{itemize}




\begin{itemize}

\item Petrucci Ralph y Harwood, William, S. Química General, 7ª ed. Prentice Hall.
\item Umland, Jean B., Bellama, Jon M. Química General, 3ª ed. International Thomson, 2000.
\item Chang, Raymond Química, 6ª ed. McGraw-Hill, México, 1999.

\end{itemize}

 
\end{document}