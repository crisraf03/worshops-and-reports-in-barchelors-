 \documentclass{article}
\usepackage[spanish]{babel}
\usepackage[utf8]{inputenc}
\usepackage[T1]{fontenc}
\usepackage{amsmath,amsfonts,amssymb}
\usepackage[dvips]{graphicx}
\usepackage[left=2cm, right=2cm, top=2cm, bottom=2cm]{geometry}
\graphicspath {{imagenes/}}
\usepackage{mathpazo}
\usepackage{float}
 \usepackage{multicol}
 
 
\title{CALORIMETRÍA: DETERMINACIÓN DEL CALOR ESPECIFICO DE UN METAL}
\author{Natali Martínez Fonseca, Giorgio Osorio Mercado, Cristian Mendoza Maldonado.\\Química General\\Química\\Facultad de Ciencias Básicas\\ Universidad del Atlántico\\grupo 1. }


\date{Martes, 17 octubre del 2017}

\begin{document}

\maketitle 

\section{Resumen}
Por medio de medición en los cambios de temperatura en una mezcla se calculó la capacidad calorífica de calorímetro y calor especifico del zinc, $7,97 \frac{cal}{C}$ y  $0.053\frac{cal}{C}) $ respectivamente.


\begin{multicols}{2}

\section{Introducción}

\begin{itemize}
    \item Calor es la energía que se intercambia entre un sistema y sus alrededores como resultado de una diferencia de temperaturas. La energía en forma de calor pasa desde el cuerpo más caliente (con una temperatura más alta) hasta el cuerpo más frio (con una temperatura más baja).
    \item Históricamente, la cantidad de calor necesaria para modificar un grado Celsius la temperatura de un gramo de agua fue llamada caloría (cal). La caloría es una unidad de energía pequeña y la unidad kilocaloría (kcal) ha sido también ampliamente utilizada. La unidad SI para el calor es simplemente la unidad SI de energía básica el julio (J).
    \item El calor especifico de una sustancia se puede determinar en un calorímetro. El calor especifico es una propiedad física intensiva de una sustancia y es la cantidad de calor (en calorías) necesario para elevar la temperatura de un gramo de la sustancia en un grado Celsius.
    \item En general, cuando una masa dada de una sustancia se somete a un cambio de temperatura, la energía térmica requerida para el cambio se da por la ecuación. 
    
    \begin{equation}
        Q=m*C_{e}* \Delta T
    \end{equation}
    
    Donde Q es el cambio en la energía de calor,  m es la masa  de la sustancia en gramos, $C_{e}$ es el calor especifico de la sustancia y $\Delta T$ es el cambio de temperatura (la diferencia entre las temperaturas final e inicial).
\end{itemize}

\section{Parte Experimental}

\subsection{Materiales}

\begin{multicols}{2}
\begin{itemize}
    \item Calorímetro de icopor.                                
    \item Termómetro.
    \item 	Pipeta volumétrica.
    \item 	Equipos de calentamiento.
    \item 	Trípode.
    \item Beaker.
    \item 	Agitador de vidrio.
    \item Zinc.
    \item Agua destilada.
    \item Vaso de precipitados
    
    
\end{itemize}
\end{multicols}


\begin{itemize}
    \item 	Capacidad calorífica del calorímetro: en un vaso de precipitados agregamos 50 ml de agua destilada pesamos para obtener la masa del agua después se calentó hasta que su temperatura fuera igual 70 °c mientras esta llegaba a la temperatura establecida en el vaso de icopor también agregamos 50 ml de agua destilada fría a 32 °c después agregamos el agua a 70 °c inmediatamente y tapamos medimos su temperatura y fue de 50 °c. 
    
    \item Calor especifico del zinc: en un vidrio de reloj pesamos 10g de zinc y lo pasamos a un tubo de ensayo después lo pusimos a un estilo de baño de maría dentro del vaso de precipitados se calentó hasta que alcanzara los 70°c mientras este llegaba a la temperatura estimada en el vaso de icopor pusimos 10 ml de agua destilada fría a 33°c, cuando nuestro metal alcanzo la temperatura designada lo agregamos rápidamente al vaso de icopor que funciona como nuestro calorímetro, tapamos agitamos un poco y pusimos el termómetro de inmediato la temperatura fue de 34°c. 
    \end{itemize}
    
\section{Resultados y Discusión}



\begin{itemize}
    \item Capacidad calorífica  del calorímetro
    
    $$q_{c}+q_{h}+q_{cal}=0$$
   \\ $$m_{a}*C_{a}*(t_{3}-t_{2})+m_{b}*C_{b}*(t_{3}-t_{1})+C_{e}(t_{3}-t_{1})$$
   \\ $$ C_{e}=\frac{(48.75)*(1 ) (70-50)-(45.99 )*(1)*(50-32)}{50-32}$$
    $$C_{e}=7.97 \frac{cal}{°C}$$
    
    Discusión: utilizamos los datos anteriores para hallar el calor especifico tenemos dos formas de trabajar con julios o calorías.
    
    \item calor especifico  del zinc
    
     $$q_{z}+q_{a}+q_{cal}=0$$
    $$m_{z}*C_{z}*(t_{3}-t_{2})+m_{a}*C_{a}*(t_{3}-t_{1})+C_{c}(t_{3}-t_{1})$$
  \\ $$C_{z}=\frac{(m_{a}*C_{a}+C_{c})*(t_{3}-t_{1})}{m_{z}*(t_{2}-t_{3})}$$
   \\$$C_{z}=\frac{(11g * 1\frac{cal}{c}+7.97 \frac{cal}{c})*(34c-33c)}{10g *(70c-34c)}$$
   $$ C_{z}= 0.053 \frac{cal}{C}$$
   
   Discusión: utilizamos el dato que obtuvimos antes para completar los calculo a este experimento reducimos los ml de agua en el vaso de icopor para que nuestro experimento nos diera de forma correcta.
\end{itemize}

\section*{Cuestionario}

\begin{itemize}
    \item ¿Que es un calorímetro?
    
    R/ El recipiente donde se realizan las experiencias en las que se producen variaciones de calor se llama calorímetro. Se trata de un recipiente que contiene el líquido en el que se va a estudiar la variación del calor y cuyas paredes y tapa deben aislarlo al máximo del exterior.
    
    \item ¿Un estudiante tiene hierro caliente y lo agrega que contiene agua fría explica que sucede en termino de flujo de calor?
    
    R/ lo que sucede es que al agregar el hierro al agua y esta esta fría comienza un flujo de calor una transferencia de calor hacia el agua hasta quedar en equilibrio. 
    
    \item ¿Por qué el agua es mejor refrigerante que el etanol?
    R/ por el calor especifico, el del agua es el más alto y es 1 esto significa que tiene mayor resistencia para que su temperatura cambie drásticamente en cambio el etanol es de menor calor especifico es de 0.586 es más débil así que se será más fácil cambiar su temperatura.
    
    \item Durante el experimento se realiza un gráfico de temperatura vs tiempo. ¿Para qué se debe realizar esto?
    R/ se debe representar la temperatura del agua en el calorímetro en función del tiempo. Puesto que las paredes del calorímetro y la cubierta no son aislantes perfectos, algo de calor se perderá al entorno, en el calorímetro algo de calor se pierde antes de que alcance la temperatura máxima se obtiene por extrapolación de la curva.
    
    \item Una muestra de 30.0g de agua se calentó desde una temperatura inicial de 18.0°c a una temperatura final de 57,5°c ¿Cuántas calorías ha absorbido el agua?
    \begin{multicols}{2}
    $$ m=30.0g$$ $$ T_{i}=18.0 C $$ $$ T_{f}=57.5 C $$ $$ Q=?$$
    
    $$Q= m*C_{e}*(T_{f}-T_{i})$$ $$Q=30.0* 1 *(57.5-18.0)$$ $$C=1185 cal$$
    \end{multicols}
    
\item Una muestra de 80,0g de esferas se calentaron en agua a 99,5°c las esferas se añadieron a 50,0g de agua que estaba a una temperatura de 20,0°c resultado en un aumento de la temperatura para el agua a 33°c.

\begin{itemize}
    \item Cual fue la temperatura a la cual se equilibraron las esferas?
    R/ 33°c
    \item Cual fue la temperatura de cambio de las esferas?
    R/ 99,5°c a 33°c
    \item Cuál es el calor especifico de las esferas?
    
    
    $$q_{agua}+q_{esferas}=0$$
    \[m_{agua}*C_{agua}*(\Delta T_{agua})+ m_{esfera}*C_{esfera}*(\Delta T_{esfera})=0\]
   
    \[C_{esfera}=\frac{- m_{agua}*C_{agua}*(\Delta T_{agua})}{m_{esfera}* (\Delta T_{esfera})}\]
    
    \[C_{esferas}=\frac{50*1*(33-20)}{80*(33-99,5)}=0,12 \frac{cal}{g*C}\]
    
    \item Empleando la siguiente tabla
    
    
    Cual seria la identidad del las esferas?
    \R Vidrio
\end{itemize}
    
    
    \item Un estudiante necesita calcular la capacidad calorifica de un calorimetro y obtuvo los siguientes resultados
    
    \begin{itemize}
        \item temperatura inicial de 60 ml de agua fría en un calorímetro $T_{i} $=24,5°c.
\item temperatura inicial de 60 ml de agua caliente $T_{i} $ = 58,5°c.
\item máxima temperatura del agua después de agregar agua caliente al agua fría en el calorímetro $T_{f}$ = 41,0°c.

    \end{itemize}

\end{itemize}



\section*{Conclusión}
El motivo de esta práctica fue calcular la capacidad calorífica de calorímetro y de un metal. obtuvimos la capacidad calorífica del calorímetro y la utilizamos para halla la del metal pudimos saber cuál fue el calor ganado y perdido de cada sustancia. En el segundo tuvimos que disminuir el agua fría de calorímetro así nos daría un valor preciso y que era adecuado a nuestra practica de laboratorio aparte de esto hallamos la caloría con referencia al calorímetro con el agua, la diferencia esto está en la hoja del reporte. Esta práctica depende más de los factores que están a su alrededor.

 
 

\end{multicols}

\begin{itemize}
 
 \item Petrucci Ralph y Harwood, William, S. Química General, 7ª ed. Prentice Hall.
 \item Umland, Jean B., Bellama, Jon M. Química General, 3ª ed. International Thomson, 2000.
 \item Chang, Raymond Química, 6ª ed. McGraw-Hill, México, 1999.
 \end{itemize}
\end{document}}